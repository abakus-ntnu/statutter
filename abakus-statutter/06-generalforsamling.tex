\section{Generalforsamling}
\subsection{Om generalforsamlingen, innkallinger og frister}
\subsubsection{}
Generalforsamlingen er Abakus sitt høyeste organ og skal avholdes minst \textbf{to} ganger i året. 
Ordinær generalforsamling skal finne sted i siste uka av februar, mens sekundær generalforsamling 
skal finne sted i siste uka av oktober. Begge kan flyttes opp til \textbf{to} uker hvis Hovedstyret ser dette som hensiktsmessig.

\subsubsection{}
På ordinær generalforsamling skal følgende skje: 
\begin{enumerate}[label=\alph*)]
    \item Regnskapet for siste regnskapsår legges frem til foreningens godkjennelse.
    \item Sittende Hovedstyret legger fram en redegjørelse for foreningens aktiviteter siden siste ordinære generalforsamling.
    \item Alle stillinger i Hovedstyret skal velges.
\end{enumerate}

På sekundær generalforsamling skal følgende skje:
\begin{enumerate}[label=\alph*)]
    \item Halvparten av stillingene i Fondsstyret skal velges.
\end{enumerate}

\subsubsection{}
Ekstraordinær generalforsamling skal innkalles når Hovedstyret eller minst
\textbf{ti} av foreningens medlemmer krever det.

\subsubsection{}
Disse fristene gjelder for generalforsamling:
\begin{enumerate}[label=\alph*)]
    \item Enhver generalforsamling skal publiseres på foreningens nettside og promoteres for minst \textbf{én måned} i forveien.
    \item Hovedstyret skal i forkant av ordinær generalforsamling lyse ut alle stillinger i Hovedstyret og 
    Fondsstyret innad i Abakus minst \textbf{én måned} før generalforsamlingen. 
    \item Søknad fra interesserte kandidater til Hovedstyret eller Fondsstyret skal være Hovedstyret 
    i hende \textbf{to uker} før generalforsamlingen.
    \item Forslag til statuttendringer eller opprettelse av komiteer skal være Hovedstyret i 
    hende minst \textbf{to uker} før generalforsamlingen. 
    \item Saksliste og sakspapirer skal være tilgjengelig for foreningens medlemmer minst \textbf{én uke} 
    før generalforsamling. Dette innebærer publisering på foreningens nettside.
\end{enumerate}

\subsection{Beslutningsdyktighet}
\subsubsection{}
Generalforsamlingen er beslutningsdyktig dersom \textbf{30} eller flere av foreningens medlemmer er til stede. 

\subsubsection{}
Alle foreningens medlemmer har stemme- og talerett ved generalforsamling. Ethvert medlem kan kreve hemmelig votering. 
Ved stemmegivning tillates ikke bruk av fullmakt. 

\subsubsection{}
Dersom ikke noe annet er nevnt i statuttene, gjennomføres alle voteringer med alminnelig flertall under generalforsamling.

\subsection{Statuttendringer}
\subsubsection{}
Statuttendringer krever kvalifisert flertall ved en generalforsamling.

\subsubsection{}
Vedtatte statuttendringer trer i kraft i det generalforsamlingen heves. Generalforsamlingen kan ved kvalifisert 
flertall vedta at en statuttendring trer i kraft umiddelbart dersom det er hensiktsmessig.

\subsubsection{}
Hovedstyret har til enhver tid myndighet til å gjøre redaksjonelle endringer (jf. \ref{def:redaksjonelle_endringer}) i
 Abakus sine statutter. Eventuelle endringer må informeres om under neste generalforsamling.

\subsection{Valg}
\subsubsection{}\label{subsec:genfors_valg}
Samtlige stillinger i Hovedstyret velges under en generalforsamling.
\begin{enumerate}[label=\alph*)]
    \item Valget avholdes med preferansevalg (jf. \ref{def:preferansevalg}). 
    \item Kandidater til Hovedstyret skal ha vært medlem i Abakus i minst \textbf{tre måneder} i forkant av valget. 
    \item Kandidater kan stille til valg under generalforsamlingen frem til valget er i gang. 
\end{enumerate}

\subsubsection{}
Det nye Hovedstyret trer i kraft fra og med \textbf{én måned} etter det blir valgt, men datoen kan 
flyttes opp til \textbf{to uker} hvis det sittende Hovedstyret ser dette som hensiktsmessig.

\subsection{Øvrig om generalforsamlingen}
\subsubsection{}
Hovedstyret skal legge frem referat fra generalforsamlingen.

\subsubsection{}
Bare generalforsamlingen kan opprette og nedlegge komiteer samt endre etiske retningslinjer.
