\section{Økonomi}

\subsection{}
Regnskapsåret i Abakus går over ett -1- år, fra og med 1.januar til og med 31. desember.

\subsection{}
\subsubsection{}
Budsjettet for det kommende året må fastsettes innen utgangen av det inneværende året. Budsjettet må godkjennes på et budsjettmøte.

\subsubsection{}
Minst én representant fra hver komité samt leder og økonomiansvarlig i Hovedstyret må være tilstede for at budsjettmøtet er beslutningsdyktig.

\subsection{}
Hovedstyret skal disponere foreningens midler i samsvar med statuttenes
paragrafer §2 og §10.

\subsection{}
Hovedstyret skal hvert år sette over et fast beløp til Abakus' fond. Beløpet
beregnes på grunnlag av Abakus' fritt disponible inntekter. Det vil si
inntekter som ikke er øremerket til et spesielt formål eller i de tilfellene
der inntektene ikke er direkte forbundet med en kostnad.

Alle fritt disponible inntekter som overskrider én million -1 000 000- kroner
skal overføres til fondet.

Beløpet skal overføres før fondets ordinære generalforsamling når regnskapene
er ferdig revidert. Eventuelt overskudd skal overføres til fondet, dersom
Hovedstyret finner det hensiktsmessig.

\subsection{}
Hovedstyret har ikke myndighet til å godkjenne søknader over 10 000 kr utenfor
oppsatt budsjett.
