\section{Statutter}

\subsection{}
For endring i Abakus’ fonds statutter kreves kvalifisert flertall ved en
generalforsamling. Alle medlemmer av Abakus kan komme med endringsforslag.

\subsection{}
Vedtatte statuttendringer trer i kraft i det generalforsamlingen heves.

\subsection{}
Fondstyret og Hovedstyret i Abakus har til enhver tid myndighet til å endre oppsett, utforming og formuleringer samt rette skrivefeil i Abakus' fonds statutter så fremt disse ikke endrer innholdet og betydningen i de aktuelle statuttene. Eventuelle endringsforslag må offentliggjøres 2. april eller 1. oktober.

Fondstyret eller Hovedstyret må redegjøre for de endringene de har gjort ved å offentliggjøre disse på foreningens nettside. Dersom et medlem av Abakus gir skriftlig uttrykk til Fondstyret eller Hovedstyret innen to -2- uker om at en endring ikke ivaretar statuttens opprinnelige intensjon, må endringen behandles på generalforsamling. Endringer som ikke blir disputert i løpet av denne perioden trer i kraft to -2- uker etter offentliggjøringen.
