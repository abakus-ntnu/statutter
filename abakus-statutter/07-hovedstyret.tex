\section{Hovedstyret}

\subsection{}
Hovedstyret er Abakus’ nest øverste organ og består av \textbf{seks} medlemmer som skal ha følgende stillinger 

\begin{enumerate}[label=\alph*)]
    \item Leder
    \item Nestleder
    \item Økonomiansvarlig
    \item Tre øvrige styremedlemmer
\end{enumerate}
Hovedstyret har ansvar fsor å drifte foreningen forsvarlig, etter generalforsamlingens vedtak.
\subsection{}
Nestleder er leders stedfortreder i leders fravær. Leder og et annet styremedlem innehar signaturrett i fellesskap. 
Alle medlemmer av Hovedstyret, samt komiteledere og revysjef, innehar individuelle prokura for sitt ansvarsområde. 

\subsection{}
Vedtak som fattes på Hovedstyrets møter krever absolutt 50 \% flertall. I
tilfelle stemmelikhet har foreningens leder dobbeltstemme.
x
\subsection{}
Hovedstyret skal føre referater fra alle sine styremøter. 

\subsection{Abakus’ fond}

\subsection{}
Abakus skal ha et fond adskilt fra Hovedstyret. Fondet heter Abakus’ fond og forvaltes av Fondsstyret.

\subsection{}
Abakus’ fonds formål er å gi Abakus en økonomisk trygghet og sørge for at Abakus sitt
økonomiske overskudd fordeles rettferdig på nåværende og fremtidige medlemmer av Abakus.

\subsection{}
Fondsstyret består av et ordensmedlem, to medlemmer som enten har vært tidligere medlem av
Hovedstyret eller tidligere komitéleder og tre Abakus-medlemmer. Ingen av Fondsstyrets medlemmer kan ha
aktive verv i Hovedstyret eller være sittende komitéleder. To av de tre Abakus-medlemmene skal ikke være 
medlemmer som innehar tittelen æresmedlem, ordensmedlem, tidligere komitéleder eller tidligere Hovedstyre-medlem.

\subsection{}
Valg til Fondsstyret gjennomføres på samme måte som valg til medlemmer av Hovedstyret (ref. § 8.4).

\subsection{}
Vervet i Fondsstyret varer to år. Tre av stillingene er til valg hver sekundære generalforsamling.

\subsection{}
En oppløsning av fondet kan kun skje dersom et enstemmig fondstyre og generalforsamlingen
ved kvalifisert flertall stemmer for. Ved oppløsning skal fondets investeringer selges, og stå
uberørt på en høyrentekonto i tre -3- år, dette for å oppfordre til gjenopptak av fondet. Dersom
det går tre -3- år etter oppløsningen uten at fondet blir gjenopptatt, tilfaller fondets midler
Hovedstyret.

\subsection{Kasserere}

\subsection{}
Kasserer i den enkelte komité er ansvarlig for den økonomiske styringen til
komiteen, og sørger for at komiteen holder seg innenfor de økonomiske rammene satt av budsjettet.
Vedkommende sitter som ansvarlig for at driften er forsvarlig.

\subsection{}
Hvis kasserer i en komité avslutter sitt verv før regnskapsårets slutt, skal Hovedstyret sørge for 
at regnskapet sluttføres og godkjenne det før påtroppende kasserer overtar.

\subsection{}
Alle kasserere og medlemmer av Bankkom må underskrive en bindende kassereravtale med foreningen 
før tiltredelse i vervet.

\subsection{}
Leder av Bankkom fungerer som vara for økonomiansvarlig i Hovedstyret.

\subsection{Undergrupper og interessegrupper}

\subsection{}
Undergrupper er økonomisk og administrativt underlagt Hovedstyret. Nestleder i
Hovedstyret har ansvaret for gruppene. Undergruppene administrerer sine egne
opptak.

\subsection{}
Interessegrupper er økonomisk og administrativt underlagt backup. Medlemskap i
interessegruppene skal være tilgjengelig for alle Abakus-medlemmer.

\subsection{Abakusrevyen}

\subsection{}
Abakusrevyen har et eget styre, og administrerer opptak og økonomi selv. Budsjett må godkjennes 
under budsjettmøtet (jf. §10.2.1).

\subsection{}
Abakus står økonomisk ansvarlig for revyen.
